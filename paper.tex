\documentclass{article}
\usepackage[utf8]{inputenc}
\title{Your Research Title}
\author{Author Name}
\date{\today}

\begin{document}
\maketitle

\begin{abstract}
A brief summary of the paper.
\end{abstract}

\section{Introduction}
Introduce the context and motivation of your research.

\section{Related Work}
Describe previously published work relevant to your paper.

\section{Methodology}
Detail your research approach and methods.

\section{Experiments}
Provide experiments, results, and analysis.

\section{Code Examples}
This appendix accompanies the repository and describes how to build and
execute the included examples written in C, Python, and x86\textendash64
assembly. Each program prints a greeting using an optional name
parameter supplied on the command line.

\subsection{C Program}
The source file \verb|code/hello.c| may be compiled using GCC as
follows:
\begin{verbatim}
gcc -o hello_c code/hello.c
\end{verbatim}
Running \verb|./hello_c| prints \texttt{``Hello, world!''}. Supplying a
name argument, e.g. \verb|./hello_c Alice|, results in the output
\texttt{``Hello, Alice!''}.

\subsection{Python Program}
The script \verb|code/hello.py| requires Python~3 and is executed with:
\begin{verbatim}
python3 code/hello.py [name]
\end{verbatim}
If a name is provided, the program prints a personalized greeting; otherwise,
it defaults to \texttt{``Hello, world!''}.

\subsection{Assembly Program}
An x86\textendash64 NASM example is provided in \verb|code/hello.asm|.
It can be assembled and linked with:
\begin{verbatim}
nasm -felf64 code/hello.asm
gcc hello.o -o hello_asm
\end{verbatim}
Executing \verb|./hello_asm Bob| prints the message
\texttt{``Hello, Bob!''}. If no parameter is supplied, the default output is
\texttt{``Hello, world!''}.

\section{Conclusion}
Summarize your findings and future work.

\bibliographystyle{plain}
\bibliography{references}

\end{document}
